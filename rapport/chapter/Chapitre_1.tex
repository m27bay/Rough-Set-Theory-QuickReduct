\section{\rst}
\rst ou la theorie des ensembles
approximatifs est basé sur un ensemble de définition que nous allons
détaillées dans ce chapitre.

\subsection{Un système d'information}
Commençons par la base : un système d'information. Un système
d'information (\textit{Information System - IS} en englais) est
une table ou une matrice avec une ligne par objets renseignés
(l'ensemble des objets est appelé l'univers (U)) et x colonnes,
appelé des attributs, décrivant les propriétés de l'objet
(l'ensemble des attributs est appelé A). \\

C'est données sont utilisées pour faire de la prise de
dcision, il nous faut pour cela un attribut spécifique,
que nous utiliserons comme décision. C'est pour cela
que nous avons créés les systèmes d'informations.

\subsection{Un système de décision}
Un système de décision (\textit{Decision System - DS}
en englais) est un système
d'information avec une, comme dit précédemment,
colonne supplémentaire appelée décision
(l'ensemble des décision est appelé $\theta$). Elle correspond
à la classe de l'objet (exemple un patient peut avoir comme
décision/ classe "malade" ou "non malade")
\begin{table}[h!]
    \centering
    \begin{tabular}{|c||c|c|c|c|}
        \hline
        $x \in U$ & a & b & c & d \\
        \hline
        $x_1$     & 1 & 0 & 2 & 2 \\
        $x_2$     & 0 & 1 & 1 & 1 \\
        $x_3$     & 2 & 0 & 0 & 1 \\
        $x_4$     & 1 & 1 & 0 & 2 \\
        $x_5$     & 1 & 0 & 2 & 0 \\
        $x_6$     & 2 & 2 & 0 & 1 \\
        $x_7$     & 2 & 1 & 1 & 1 \\
        $x_8$     & 0 & 1 & 1 & 0 \\
        \hline
    \end{tabular}
    \caption{Exemple d'un système d'information}
    \label{tab:msysteme_information}
\end{table}

\begin{table}[h!]
    \centering
    \begin{tabular}{|c||c|c|c|c||c|}
        \hline
        $x \in U$ & a & b & c & d & e \\
        \hline
        $x_1$     & 1 & 0 & 2 & 2 & 0 \\
        $x_2$     & 0 & 1 & 1 & 1 & 2 \\
        $x_3$     & 2 & 0 & 0 & 1 & 1 \\
        $x_4$     & 1 & 1 & 0 & 2 & 2 \\
        $x_5$     & 1 & 0 & 2 & 0 & 1 \\
        $x_6$     & 2 & 2 & 0 & 1 & 1 \\
        $x_7$     & 2 & 1 & 1 & 1 & 2 \\
        $x_8$     & 0 & 1 & 1 & 0 & 1 \\
        \hline
    \end{tabular}
    \caption{Exemple d'un système de décision}
    \label{tab:msysteme_decision}
\end{table}
Dans un IS ou un DS, il est possible d'avoir de la redondance
dans les données. Ces données peuvent avoir un, deux voir même
tous leurs attributs égales. Nous pouvons créer des sous ensembles
de données grâce à ce critère, c'est ce qu'on appelle : les classes
d'équivalence.

\subsection{Classe d'équivalence}
Mathématiquement la classe d'équivalence est définit comme suit :
Soit un ensemble E muni d'une relation d'équivalence, noté $\sim$.
Ici deux objets sont équivalent si et seulement si ils possèdent
les mêmes valeurs pour chaque attributs. On définit une classe
d'équivalence, noté $[x]$, d'un élément x de E comme l'ensemble
des y de E tels que $x \sim y$. \cite{wiki_relation_equivalence}
\begin{equation}
    y \in [x] \iff x \sim y
\end{equation}

La redondance des données n'est pas forcèment problèmatique dans le
machine learning, sauf dans un cas particulier : si deux données
possèdant les mêmes attributs ont une valeur de décision différentes.
Lorsque deux objets respectent cette règle, nous avons une
\ind entre les deux.

\subsection{\ind}
Soit I = (U, A), avec I un système d'information, U un ensemble
d'objets et A un ensemble d'attributs. Avec n'importe quel sous
ensemble $P \subseteq A$. Il existe une relation d'équivalence,
noté IND(P), définit comme :
\begin{equation}
    IND(P) = \{(x, y) \in U^2 | \forall a \in P, a(x) = a(y)\}
\end{equation}
La relation IND(P) est la \ind. \\
Autrement dit deux objets sont indiscernable si pour toutes
propriétés $a \in P, a(x) = a(y)$. \\
Dans l'exemple TABLE 2, nous nous pouvons définir trois
relations indiscernables sur les attributs \{a\}, \{b, c\} et
\{a, b, c\} qui définissent les trois portions de l'univers :
\begin{equation}
    IND(a) = \{\{x_1, x_4, x_5\}, \{x_2, x_8\}, \{x_3, x_6, x_7\}\}
\end{equation}
\begin{equation}
    IND(b, c) = \{\{x_3\}, \{x_1, x_5\}, \{x_2, x_7, x_8\}, \{x_8\}\}
\end{equation}
\begin{equation}
    IND(a, b, c) = \{\{x_1, x_5\}, \{x_2, x_8\}, \{x_3\},
    \{x_4\}, \{x_6\}, \{x_7\}\}
\end{equation}
Si nous calculons la relation indiscernable sur l'ensemble des
attributs du système, nous notons cela $IND_C$ ou $U/C$.

\subsection{\textit{rough set}}
Soit un ensemble X d'objets cibles tel que $X \subseteq U$.
Nous souhaitons représenté X en utilisant un sous-ensemble
d'attributs P. X comprend une seule classe et nous voulons
décrire cette classe en à partir des classes d'équivalence des
attributs dans P. En général, X ne peut pas être décrit
précisément car il peut inclure ou exclure des objets qui
ne sont pas distingués sur la base des attributs de P.
Pour résoudre ce problème nous approchons X en le bornant avec
la \blower notée \underline{P}(X), et
la \bupper notée $\Bar{P}(X)$. \\

\newpage
\subsection{\blower}
La \blower est définit comme suit :
\begin{equation}
    \underline{B}(X) = \{o_j | [o_j]_B \subseteq X\}
\end{equation}
C'est l'ensemble complet des objets dans
$U/P$ qui peuvent être classés dans X sans ambiguïté.

\subsection{\bupper}
La \bupper est définit comme suit :
\begin{equation}
    \Bar{B}(X) = \{o_j | [o_j]_B \cap X \neq \emptyset\}
\end{equation}
C'est l'ensemble des objets dans
$U/P$ qui peuvent possiblement être classés dans X. Ce sont des objets
possèdent un objets indiscernables. \\

Grâce à ces deux approximations nous pouvons calculer \bboundary
notée $BN_B(X)$.

\subsection{\bboundary}
La \bboundary est définit comme :
\begin{equation}
    BN_B(X) = \Bar{B}(X) - \underline{B}(X)
\end{equation}

Une fois que nous avons calculer les différentes approximations pour
chaque valeurs de décisions, nous pouvons calculer les regions.

\subsection{\posreg}
La \posreg est définit comme ceci :
\begin{equation}
    POS_C\{(d)\} = \bigcup_{X \in U/\{(d)\}} \Bar{C}(X)
\end{equation}

\subsection{\negreg}
La \negreg est définit comme ceci :
\begin{equation}
    NEG_C\{(d)\} = U - \bigcup_{X \in U/\{(d)\}} \underline{C}(X)
\end{equation}

\subsection{\reduct}
\reduct est un sous ensemble minimal d'attributs ayant la même
\posreg que l'ensemble des attributs.

\subsection{\core}
\core est un ensemble d'attributs indépendant incluant tous les
\reduct. \\

\noindent
Ceci conclus l'explication du \rst. A présent nous allons détaillé
le fonctionnement des algorithmes du \rst et nous parlerons d'une
des heuristique, le \quickreduct.